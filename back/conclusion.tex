% !Mode:: "TeX:UTF-8" 
\begin{conclusions}

包括分词、词性标注、命名实体识别以及句法分析在内的
语言分析是基础而重要的自然语言处理问题。
提高语言分析的性能能够有效地帮助下游任务。
强化词表示能力是提高语言分析性能的一种有效手段。
近年来提出的上下文相关词向量已经帮助一部分自然语言处理任务
取得了性能提升。
本文受到相关研究的启发,
针对上下文相关词向量与语言分析的结合开展了一系列研究。

本文针对现有上下文相关词向量因过分强调使用多层网络对
一整句甚至多句进行抽象而导致的效率问题,
从语言分析主要依赖局部信息出发,
提出一种融合相对位置权重的窗口级自注意力机制
并将其应用于上下文表示,从而获得一种适用于语言分析的上下文相关词向量。
本文在五项语言分析任务上进行实验,
相应结果表明本文提出的上下文相关词向量在
不损失精度的情况下获得了三倍的速度提升。

基于前文提出的上下文相关词向量,
本文针对语言分析中的切分问题(中文分词与命名实体识别)
对合理的片段(词与实体)表示的依赖,
本文在上下文相关词向量的基础上
提出一种基于简单拼接的片段表示方法并
将其应用于半-马尔科夫条件随机场中。
在典型切分问题的实验中,
本文基于上下文相关词向量的片段表示有效地提高了模型性能。
通过进一步融合任务相关的上下文表示以及
建模片段级信息的片段向量,本文模型取得了当前最优的性能。

基于前文提出的上下文相关词向量,
本文针对上下文相关词向量对多国语句法分析作用尚无明确结论的现状,
提出在多国语句法分析中使用上下文相关词向量
并在大规模树库上验证上下文相关词向量的有效性。
除了获得稳定且显著的性能提升,
本文针对提升的原因进行了详细的分析。
大量分析实验表明,
性能提升的主要原因是上下文相关词向量
通过对于未登录词词形的更好的建模
有效地提升了未登录词的准确率。

针对使用上下文相关词向量的句法分析参数过多、运行速度较慢的问题,
本文提出一种结合探索机制的知识蒸馏算法,
以将基于上下文相关词向量的复杂模型
用不使用相应词向量的简单模型进行近似,
从而在不显著降低性能的情况下提高句法分析速度。
实验结果表明,
本文提出的方法
在损失少量句法分析准确率的情况下,
近十倍地提升了速度。

本文研究表明上下文相关词向量可以很好地与语法分析结合,
从而显著提高语法分析的性能。
同时,本文也以语言分析为工具
对于上下文相关词向量进行全面的分析。
这些分析表明上下文相关词向量具有表示未登录词,
建模片段的能力。
%学位论文的结论作为论文正文的最后一章单独排写,但不加章标题序号。
%
%结论应是作者在学位论文研究过程中所取得的创新性成果的概要总结,不能与摘要混为一谈。博士学位论文结论应包括论文的主要结果、创新点、展望三部分,在结论中应概括论文的核心观点,明确、客观地指出本研究内容的创新性成果(含新见解、新观点、方法创新、技术创新、理论创新),并指出今后进一步在本研究方向进行研究工作的展望与设想。对所取得的创新性成果应注意从定性和定量两方面给出科学、准确的评价,分(1)、(2)、(3)…条列出,宜用“提出了”、“建立了”等词叙述。

\end{conclusions}
