% !Mode:: "TeX:UTF-8"
\begin{acknowledgements}

本文是在导师秦兵教授与副导师车万翔教授的共同指导下完成的。
没有两位老师对于文章内容的编排以及对于文章进度的督促,本文是无法在保证质量的情况下按时完成的。
同时,刘挺教授及其领导的哈工大社会计算与信息检索研究中心
也为本文的完成提供了充分的支持。
具体到文章内容,
本文第\ref{chp:semicrf}章内容得到了麻省理工学院的郭江研究员的指导与帮助。
本文第\ref{chp:seqlabel}章的实验是在王宇轩与郑博两位师弟的共同努力下完成的。
本文第\ref{chp:distill}章的实验是在赵怀鹏师弟的帮助下完成的。
除此之外,
天津大学的张梅山副教授对于本文第\ref{chp:seqlabel}章,
文灏洋师弟对于本文第\ref{chp:elmo}与\ref{chp:seqlabel}章
分别给出了宝贵的意见。
本文的完成与各位同仁的辛勤劳动是密不可分的。

除了博士论文关注的内容,
本文作者博士期间的相关研究也得到了多位同仁的帮助指导。
其中包括苏州大学的李正华副教授、
西湖大学的张岳副教授、
美国华盛顿大学的Noah A. Smith副教授、
美国乔治华盛顿大学的Nathan Schneider助理教授、
剑桥大学的朱懿以及王少磊、侯宇泰、覃立波、文灏洋四位师弟。

自本文作者于2011年进入实验室开始科研工作,
导师秦兵教授给予本文作者多方面的关怀。
其中既有对科研的指导,也有对生活的鼓励。
这些指导与鼓励使本文作者感受到家的温暖,
更加地热爱这个集体。

自2008年考入哈工大计算机学院到2019年提交博士论文,
本文作者的成长过程也得到了副导师车万翔教授的巨大帮助。
2008年秋,车教授任教高级语言程序设计并给本文作者当年的唯一满分。
此事培养了本文作者对于计算机的兴趣与自信。
2011年,车教授介绍本文作者进入社会计算与信息检索研究中心开展科研工作,
并将实验室的重点项目 --- 语言技术平台的开发工作交付给本文作者,
使其对于语言分析技术有了深入的了解与认识。
2013年与2016年,车教授又分别推荐本文作者去新加坡、美国交流访学,
开阔了其研究思路与视野。

实验室主任刘挺教授也从多方面对本文作者进行了指导。
刘挺教授的言传身教塑造了本文作者的基本学术价值观。
同时,在本文作者开始科研工作之初,
刘挺教授强调的``一万小时''、``成为专家''
的观点都对本文作者产生了积极的影响。
刘挺教授也积极为包括本文作者在内的实验室成员创造学习交流机会。
在与毕业师兄交流中,本文作者学得的时间管理技术并受益良多。

2013年到2014年,本文作者访问新加坡科学与设计大学,
在张岳副教授的指导下进行了一年的科研工作。
这段访问经历从基本科研方法、论文写作等方面锻炼了本文作者,
对于本文作者的快速成长有重大意义。

2016年到2017年,本文作者访问美国华盛顿大学并得到Noah A. Smith教授的指导。
Smith教授帮助本文作者更好地理解了自然语言处理与机器学习之间的关系。
更重要的是,通过与Smith教授的交流,本文作者认识到学术合作的重要性。

本文作者也要感谢实验室的张宇教授,刘铭副教授,丁效老师以及冯骁骋老师。
几位老师在工作生活中对本文作者给出有效的建议。
同时,本文作者也要感谢包括牛国成、韩冰、徐伟、徐梓翔、刘洋等等共同奋斗的同学。

最后,本文作者需要感恩父母给予其生命,养育其成人,支持其完成博士学业。

%本文作者妄自感谢快速变革的时代。

\end{acknowledgements}
